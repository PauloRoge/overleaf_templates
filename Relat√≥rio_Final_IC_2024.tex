\documentclass[article,10pt,twoside,a4paper,twocolumn,english,brazil]{abntex2}

% PACOTES
\usepackage[utf8]{inputenc}
\usepackage[T1]{fontenc}
\usepackage{indentfirst}
\usepackage{color}
\usepackage{graphicx}
\usepackage{microtype}
\usepackage{soulutf8}
\usepackage{mathptmx}
\usepackage{titling}
\usepackage{etoolbox}
\usepackage{titlesec}
\usepackage{ragged2e}
\usepackage{comment}
\usepackage{float}
\usepackage{cite} % Use this package for numerical citations
\usepackage{url}
\usepackage[labelfont=bf,format=plain,justification=justified,singlelinecheck=false]{caption}
%\usepackage[alf,bibjustif,abnt-etal-list=0,abnt-etal-cite]{abntex2cite}

\usepackage{amsmath,amssymb,amsthm} 
\usepackage{graphicx}
\usepackage{lastpage}
\usepackage{multirow}
\usepackage{wrapfig}
\usepackage{url}  % pacote para quebrar linha da URL

% MARGENS ESQUERDA, DIREITA, SUPERIOR E INFERIOR
\setlrmarginsandblock{2.5cm}{2cm}{*}
\setulmarginsandblock{2cm}{2.3cm}{*}
\checkandfixthelayout

% OUTROS ESPACAMENTOS
\setlength{\droptitle}{-1.3cm}
\setlength{\columnsep}{0.5cm}
\setlength{\parindent}{0.5cm}
\setlength{\parskip}{0.0cm}

% LINHA HORIZONTAL
\def\Vhrulefill{\leavevmode\leaders\hrule height 0.7ex depth \dimexpr0.5pt-0.7ex\hfill\kern0pt}

% NOTAS DE RODAPE
\thanksmarkseries{arabic}
\setlength\thanksmarkwidth{.5em} %adicionei
\renewcommand{\footnoterule}{
    \kern -3pt
    \hrule width 4.2cm height 0.5pt
    \kern 2.6pt
}
\let\oldmakethanksmark\makethanksmark
\renewcommand{\makethanksmark}{\oldmakethanksmark\small}
\let\oldthanks\thanks
\renewcommand{\thanks}[1]{\oldthanks{\small #1}}

% SECOES E SUBSECOES
\titleformat*{\section}{\fontsize{11pt}{13.2pt}\bfseries}
\titleformat*{\subsection}{\itshape}{\fontsize{11pt}{1.2pt}}
\titleformat*{\subsubsection}{\itshape}{\fontsize{11pt}{1.0pt}}

% REFERENCIAS
\apptocmd{\thebibliography}{\justifying}{}{} 

% ----ALGUNS DESTES DADOS SERÃO ADEQUADOS PELA EDITORAÇÃO CIENTÍFICA  -----------%

% INFORMACOES DO ARTIGO E DA REVISTA

\newcommand{\paginainicial}{3} % página inicial
\newcommand{\doi}{10.5433/1679-0375.2020v41n1p3} % DOI

\newcommand{\volume}{41} % volume da revista
\newcommand{\numero}{1} % numero da revista
\newcommand{\mes}{Jan./June } % mes de publicacao da revista
\newcommand{\ano}{2020} % ano de publicacao da revista

%----------autores tem que informar os dados AQUI----------
\newcommand{\autorcabecalho}{%Cândido, P.~R.
} % nomes dos autores no cabecalho
\newcommand{\titulocabecalho}{title in english} % titulo no cabecalho

%-----EDITORAÇÃO CIENTÍFICA  incluirá estas informações ------
\newcommand{\datarecebidoen}{Aug. 30, 201?} % data de recebimento do artigo em ingles
\newcommand{\dataaceitoen}{Jan. 15, 202?} % data de aceite do artigo em ingles

%---------------------------------------------------------%

% PAGINA INICIAL
\setcounter{page}{\paginainicial}

% CABECALHO E RODAPE DA PRIMEIRA PAGINA
\makepagestyle{firstpage}
\makeoddhead{firstpage}{\fontfamily{phv}\scriptsize\selectfont ORIGINAL ARTICLE \\[-0.1cm] DOI: \doi}{}{}
\makeoddfoot{firstpage}{}{\fontfamily{phv} \footnotesize \selectfont \hspace{1cm} \Vhrulefill \hspace{0.65cm} \thepage \\ Universidade Estadual de Londrina, Londrina, v. \volume, n. \numero, p. \paginainicial-\pageref{LastPage}, \mes \ano}{}

% CABECALHO E RODAPE DAS DEMAIS PAGINAS
\makepagestyle{text}
\makeevenhead{text}{}{\fontfamily{phv} \footnotesize \selectfont \autorcabecalho}{}
\makeoddhead{text}{}{\fontfamily{phv} \footnotesize \selectfont \titulocabecalho}{}
\makeevenfoot{text}{}{\fontfamily{phv} \footnotesize \selectfont \thepage \hspace{0.65cm} \Vhrulefill \hspace{1cm}~\\ Universidade Estadual de Londrina, Londrina, v. \volume, n. \numero, p. \paginainicial-\pageref{LastPage}, \mes \ano}{}
\makeoddfoot{text}{}{\fontfamily{phv} \footnotesize \selectfont \hspace{1cm} \Vhrulefill \hspace{0.65cm} \thepage \\ Semina: Ciências Exatas e Tecnológicas, Londrina, v. \volume, n. \numero, p. \paginainicial-\pageref{LastPage}, \mes \ano}{}

% TITULO % Gustavo 19 maio
\titulo{ \fontsize{16pt}{19.2pt} \selectfont
    \textit{\textbf{ISAC - Sensing and Communication Integration MIMO
}} % titulo em inglês
}
\tituloestrangeiro{\fontsize{16pt}{19.2pt} \selectfont
    \textbf{}  % titulo em português
    \vspace{0.3cm}
}

% AUTORES: Nome completo. Sobre os demais dados use abreveaturas, se necessário, para manter todas as informações/descrições,  de cada autor, em apenas uma  linha.   
\autor{ \fontsize{13pt}{15.6pt} \selectfont
   Paulo Rogerio Alfredo Candido Junior\thanks{Universidade Estadual de Londrina, Londrina, Paraná, Brasil, E-mail: paulo.rogerio.candido@uel.com}; % primeiro autor e sua descricao
    %Author's full name\thanks{Prof. Dr., Depto. ????, Instituição, Cidade, Estado, País, E-mail: author2@gmail.com}; % segundo autor e sua descricao
    %Author's full name\thanks{Prof. Ms., Depto. ????, Instituição, Cidade, Estado, País, E-mail: author3@gmail.com}; % terceiro autor e sua descricao
    %\\ \fontsize{13pt}{15.6pt} \selectfont Author's full name\thanks{Dra., Depto. ????, Instituição, Cidade, Estado, País, E-mail: author4@gmail.com}\\
        %\colorbox[rgb]{1,1,0}{autores enviar  informações completas - Mini currículo em português}
   % Author 5\thanks{ Master student, Dept. Chemistry, USP, SP, Brazil; E-mail: author5@gmail.com}
}

% DATA
\data{}

\begin{document}

\captionsetup{format=plain,justification=justified,singlelinecheck=false} 

% DEFINICOES DO ARTIGO E PRE-TEXTUAIS
\selectlanguage{english}
\frenchspacing

\pretextual
\SingleSpacing
\onecolumn

% TITULO E AUTORES
\maketitle
\vspace{-1.2cm}

% ESTILO DA PRIMEIRA PAGINA
\thispagestyle{firstpage}

% RESUMO E PALAVRAS-CHAVE EM INGLES % Gustavo 19 maio
\renewcommand{\resumoname}{\texorpdfstring{\fontsize{14pt}{16.8pt} \fontfamily{ptm} \selectfont \textbf{Abstract} \\[-0.3cm] \rule{0.91\textwidth}{0.5pt} \vspace{-0.5cm}}{Abstract}}
\begin{resumoumacoluna}
    \normalsize
	This is the template to edit the article to be submitted to Semina: Exact and Technological Sciences. An information summary of approximately 200 words in \textbf{English} should be included here. This should be informative and not only indicate the general scope of the article but also state the main results obtained, methods used, the value of the work and the conclusions.
Following the keywords, maximum five (5), in English separated by dot, always with the first letter in Upper case.
	
	\vspace{0.15cm}
    \noindent \textbf{Keywords:} Keyword 1. Keyword 2. Keyword 3. Keyword 4. Keyword 5. 	 
	
\end{resumoumacoluna}
\vspace{0.3cm}

% RESUMO E PALAVRAS-CHAVE EM PORTUGUES % Gustavo 19 maio
\renewcommand{\resumoname}{\texorpdfstring{\fontsize{14pt}{16.8pt} \fontfamily{ptm} \selectfont \textbf{Resumo} \\[-0.3cm] \rule{0.91\textwidth}{0.5pt} \vspace{-0.5cm}}{Resumo}}
\begin{resumoumacoluna}
    \normalsize
    \begin{otherlanguage*}{brazil}
	O Projeto de iniciação científica tem como objetivo explorar diversas técnicas computacionais, para melhorar sistemas de comunicação sem fio 5G, focando em sistemas MIMO (multiple input multiple output) de larga escala com um elevado número de antenas (N $\geq$ 64). Busca analisar e caracterizar métodos estatísticos e de aprendizado de máquina para resolver problemas clássicos em massive MIMO, como detecção, estimação de canal, localização, sensing, alocação de recursos, incluindo espectro-tempo (ou sub-canais) e potência de transmissão. Através de simulações computacionais e métodos estatísticos, para identificar e aplicar as técnicas mais eficazes vizando otimizar o desempenho dos sitemas 5G em termos  do espectro, robustez da comunicação e precisão na detecção e estimação de canais.
         
				\vspace{0.15cm}
        \noindent \textbf{Palavras-chave:} Palavra-chave 1. Palavra-chave 2. Palavra-chave  3. Palavra-chave  4. Palavra-chave  5.
	
     
    \end{otherlanguage*}
\end{resumoumacoluna}
\vspace{0.3cm}

%---------------------------------------------------------%

% DEFINICOES TEXTUAIS
\textual
\OnehalfSpacing
\twocolumn
\pagestyle{text}

% SECOES E SUBSECOES
\section*{Introduction}

% Exemplo de citação
Esta é uma citação de exemplo.

%---------------------------------------------------------%

% DEFINICOES POS-TEXTUAIS
\postextual
\pagestyle{text}

% AGRADECIMENTOS
\section*{Acknowledgment}
 The authors wish to thank the Laboratory of Microscopy and Microanalysis (LMEM) and Laboratory of Spectroscopy (ESPEC) of State University of Londrina, CAPES-DS and CNPq (N$^{\circ}$ 479768/2012-9) - Brazil, for financial support.

% REFERENCIAS
\section*{References}
\vspace{-1.5cm}
\renewcommand{\bibname}{}
%%%%%%%%%%%%%%%%%%%%%%%%%%%%%%%%%%
\providecommand{\abntreprintinfo}[1]{%
 \citeonline{#1}}
\setlength{\labelsep}{0pt}

\providecommand{\abntrefinfo}[3]{}
\providecommand{\abntbstabout}[1]{}
\abntbstabout{v-1.9.6 }

%%%%%%%%%%%%%%%%%%%%%%%%%%%%%%%%%%%%%%

%Bibliographical citations should follow the system of alphabetical order (NBR 10520 of ABNT). 
\bibliographystyle{IEEEtran} 
\bibliography{Refs_pibiti.bib}


%---------------------------------------------------------%

% DATAS DE RECEBIMENTO E DE ACEITE DO ARTIGO
\newpage % descomentar esta linha se as datas estiverem na primeira coluna
\vspace*{\fill}

\end{document}
